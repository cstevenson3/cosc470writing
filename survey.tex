\documentclass[acmsmall, 12pt]{acmart}

\begin{document}

\title{A Survey of Medical Imaging Reconstructions and Rendering}
\author{Cameron Stevenson}
\affiliation{
 \institution{University of Canterbury}
 \streetaddress{Ilam}
 \city{Christchurch}
 \state{Canterbury}
 \country{New Zealand}}
 
\begin{abstract}
To understand the approaches and progress of creating reconstructions/renders from medical imaging, I surveyed papers involving applications, image processing, reconstructions, rendering, and testing. It was found that many approaches and specific techniques are available, and tweaking, modifying, or combining these existing methods could be a starting step for research into reconstructions/renders of particular body structures.
\end{abstract}

\maketitle

\section{Introduction}

Medical imaging modalities such as HRCT and MRI produce image stacks depicting regions of different tissue. These stacks can be used to produce 3D reconstructions and renders of internal body structures. These models and renders are used in applications such as diagnosis, treatment, education, surgical simulation, and robot assisted surgery \cite{mackay2019robust, mukundan2016reconstruction, pan2017comparison}. The process of creating 3D reconstructions or renders can be improved by automation (reducing user input), making output models/renders more accurate, or improving runtime performance.

\section{Overview}

Images from scans (also referred to as sections or slices) can be segmented based on intensity into pixel regions to define structure boundaries. These can be processed further by finding contours to represent these boundaries.

Any medical imaging reconstruction/render is typically concerned with a particular area of the body, dependent on the application. All applications tend to use either volumetric rendering or surface rendering.

Volumetric rendering treats pixels in images as voxels, and a variety of rasterization and raycasting techniques are available for rendering these.

Surface rendering requires a surface be defined, either implicitly or as a mesh. Point cloud methods generate implicit surfaces, whilst meshes can be generated through contour correspondence followed by mesh triangulation (with point correspondence as an optional middle step). 

\section{Applications}

Pan et al. \cite{pan2017comparison} refers to the importance of 3D reconstructions and rendering in robot-assisted surgery. With real time rendering being a priority, surface reconstructions are preferred over volume rendering. Of the four methods analysed, marching cubes was found to be the most suitable.

Lim et al \cite{lim2016use} found that the use of 3D printed cardiac models resulted in a statistically significant improvement in test scores of medical students.

\section{Image Processing}

As with most areas involving measurement of the real world, unwanted artefacts such as noise are introduced in the imaging process. There can also be variation from subject to subject and between imaging machines. Therefore standard image processing techniques are used to pick out the parts of images which are relevant to the application. Running signal processing on the 2D data (as opposed to considering the whole image stack in the reconstruction stage) reduces complexity. However techniques considering the image data from all slices have been considered.

\subsection{Segmentation}

Segmentation is the process of picking out the pixels belonging to individual objects in an image. From this the projected geometry of the object onto the image can be inferred, which is then used in reconstruction or rendering. Birkfellner \cite{birkfellner2016applied} observes that organs are usually composed of multiple tissue types, which show up as different intensities under imaging. This makes segmentation "a rather complex, specialized procedure often requiring considerable manual interaction". Particular organs are often focussed on when developing segmentation methods.

Birkfellner \cite{birkfellner2016applied} covers some advanced segmentation methods. The watershed transform for example uses the physical idea of water running to the bottom of valleys in a landscape. After taking a gradient transform on an image, edges are peaks in the landscape, and the virtual water will fill up basins representing segments in the image. Various interpretations of the physical behaviour can be used.

Carr \cite{carr1996surface} refers to various morphological methods used to remove noise. An opening operation acts like a low pass filter whilst still preserving edges. Opening and closing in sequence tends to be better at maintaining the mean intensity.

Mukundan \cite{mukundan2016reconstruction} observes that in HRCT lung scans, tissue regions are "characterized by different and easily separable intensity levels". In this case simple thresholding can be used to pick out regions. 

\subsection{Contour Finding}

Rather than use pixels/voxels, some reconstruction techniques use contours defining the boundaries of the region objects occupy in an image. 

Mukundan \cite{mukundan2016reconstruction} starts with a binary image after thresholding. Eroding the image with a 3x3 element then subtracting this from the thresholded image gives one pixel wide edges. Sequential edge following is used to extract contours. Discarding small contours reduces the number of contours significantly.

Pu et al. \cite{pu2008adaptive} introduce a border marching algorithm with an adaptive step size to find the outer contours of the lungs. The metric for adjusting the step size for a border segment is based on how far (at most) the segment lies from the true border. This method has the advantage of including small juxtapleural pulmonary modules in the segmentation despite their imaged intensity being dissimilar to the rest of the lung. Mackay \cite{mackay2019robust} uses this method.

\subsection{Contour Interpolation}

Between two slices filled with contours, new slices can be added with contours interpolated from those in the slices above and below them. Some methods are able to do this without a contour correspondence.

Barrett et al. \cite{barrett1994image} present a contour interpolation algorithm in image space based on morphological operations. An image with both contours present (as different grayscale values) is dilated until the space between contours is filled. The front where the two dilations meet is where the interpolated contour is found. It is noted that this method handles branching cases with no modification necessary.

\section{Volumetric Rendering}

Each image in the image stack is treated as a slice with thickness. Thus the pixels in the image are voxels, and various volume rendering techniques can be used to directly render the data without an intermediate structural representation such as a mesh. Since every voxel may be involved in the final render, naive implementations can be expensive. Techniques to improve the render quality and performance have been investigated.

\subsection{Rasterization}

In rasterization, the forward direction of an object's effect on an image is considered. Each voxel may directly affect the final render, or first be projected onto a intermediate object which is then itself drawn.

Splatting takes each voxel's value and "splats" it against the drawn image, contributing to a few pixels, with its contribution fading away as you move outwards. Zwicker et al. \cite{zwicker2002ewa} use elliptic Gaussian kernels as the basis of the shape of each splay.

Texture-based volume rendering intersects many planes with the volume \cite{ohiotexture}. On these planes polygons are rendered, with texture mappings from their coordinates to the 3D space of the volume, to pull texture values from the voxels. The planes must not be parallel to the viewing direction.

\subsection{Raycasting}

In raycasting, we take each pixel of the render and work backwards to find which objects affect it. Each pixel emits a ray which intersects with many voxels. The weighting of voxels is dependent on the technique.

Maximum intensity projection (MIP) is a raycasting method where rays project the most intense voxel they pass through \cite{birkfellner2016applied}. The images produced have high contrast detail and are easy to understand. Summed voxel rendering is another raycasting method where rays sum up intensities from every voxel they pass through, giving a blurred image \cite{birkfellner2016applied}.

Fishman et al. \cite{fishman2006volume} make comparisons between maximum intensity projection and other volume rendering. MIP tends to not contrast the background well with the structure of interest. Other volume rendering methods can weight voxels differently and give different tissue types different colours. 

Intersecting arbitrary rays with voxels can be computationally expensive. Shear-warp rendering solves this by using projections which make rays orthogonal to the voxel axes \cite{lacroute1994fast}. 

\section{Surface Reconstructions}

In each image of the image stack, we can see where the boundaries of tissue are. We can therefore describe the geometry of a structure as it intersects the image plane. By combining all images in the stack, a surface reconstruction of the entire object can be found, provided the relationships between slices are inferred accurately.

\subsection{Marching Cubes}

Marching cubes \cite{lorensen1987marching} converts voxels into surfaces. Each voxel either belongs to a structure or does not, based on imaged intensity. Surface voxels ("inside" voxels bordering "outside" voxels) are found. Each voxel is assigned a set of triangles based on which of their neighbours are inside or outside the structure, using a lookup table. The triangles are joined in neighbouring surface voxels to form the overall mesh. The results tend to look jagged and smoothing is usually applied either to the mesh or during rendering for a more visually appealing output.

Newman et al. \cite{newman2006survey} have conducted a comprehensive survey of marching cubes. One problem marching cubes has is ambiguity which induces defects. The survey recommends avoiding lookup tables with reflective symmetry, where "outside" and "inside" corners of a voxel are swapped, giving the same triangle set but with opposite normals. Instead only rotational symmetry (rotating corners around the voxel) is allowed. Such a lookup table was identified by Nielson et al. \cite{nielson2003marching}.

\subsection{Point Cloud Methods}

One of the more general ways of defining a surface with incomplete data is by sampling many points from it, and making assumptions about the way these points would be connected. This is common in 3D depth scanning of exteriors of objects, where many points are sampled but the entire surface is not known and must be inferred. If we treat the boundaries of tissue regions in images as sets of points, we can apply point cloud methods to all the points gathered. The techniques can be tuned for specific applications. Methods which approximate produce surfaces leaning more heavily on the assumptions made with the points guiding the end results. This is useful when the sampling of the points is noisy. Methods which interpolate assume the points are perfect and so the surfaces produced must pass through them. It is common to define a function from 3D space to a value so that the surface should be found where the function output is zero, then an isosurface is generated.

\subsubsection{Models}

Braude et al. \cite{braude2007contour} employ Multi-level Partition of Unity (MPU) implicit models to generate isosurfaces from. MPU closely approximates Euclidean distance near points. This method requires surface normals. 

Guennebaud et al. \cite{guennebaud2007algebraic} fit algebraic spheres to point sets to construct surfaces. Their method (APSS) performs better than prior methods on sharp features and sparse data.

Oztireli et al. \cite{oztireli2009feature} combine Moving Least Squares (MLS) with local kernel regression to obtain Robust Implicit Moving Least Squares (RIMLS). This method reconstructs sharp (non-smooth) corners more accurately than APSS.

Taubin et al. \cite{taubin2012smooth} demonstrate colour maps extrapolated from source points onto a reconstructed surface.

\subsubsection{Estimating Normals}

Some of the methods above require normals for each point. Estimating normals from the points sampled requires some understanding of the structure itself.

Mitra et al. \cite{mitra2003estimating} use least squares distance to fit a plane to a neighbourhood of points for each point in the cloud. There is a sweet spot for the radius of the neighbourhood used. Small radius makes noisy points have more impact on the plane found, and large radius allows for surface curvature to introduce error. 

\subsection{Contour Correspondence}

Rather than lose the relationship between points, the original contours can be used in creating a surface reconstruction. Working out which contours on which slices represent a connected tissue in the original structure is called contour correspondence, and the resulting correspondences are used in later steps of reconstruction.

Herbert et al. \cite{herbert2001contour} classify correspondence algorithms into four types.
\begin{itemize}
\item Manual methods use user input to connect contours. This is time consuming for large datasets.
\item Local algorithms takes pairs of slices at a time and considers contour matchings between these.
\item Global algorithms look for contour pairings across all sections.
\item Growing algorithms create a hierarchy of components, attempting to join unmatched contours onto existing components if suitable.
\end{itemize}

Herbert et al. \cite{herbert2001contour} suggest growing objects one contour at a time instead of considering pairings of contours globally. In preprocessing spacial information is gathered such as contour characteristics (position, shape, size), intra-sectional relationships between contours (to validate complex structures later), and inter-sectional relationships. Contour relationship metrics include distance between centroids, distance between major/minor axes, minimum bounding rectangle overlap, shape comparison via compactness ratios, and surroundness (how deep is the contour nested in larger contours).

Herbert et al. \cite{herbert2001contour} include semantic information given by the user on the expected components in a reconstruction, and their spatial relation to each other. A starting contour is found for each component before the growing process starts.

\subsection{Point Correspondence}
Point correspondence is an optional step in surface reconstructions, where points on matched contours are matched to each other as a precursor to triangulation.

Mackay \cite{mackay2019robust} proposes Dynamic Time Warping (DTW) as a method of point correspondence. DTW is intended to match features on the same structure across different times. In point correspondence, it matches points on contours which are from the same structure but in different slices, so slightly warped. 

\subsection{Mesh Triangulation}

A surface can be represented as a set of primitive elements, with triangles being the simplest. They are defined by three points in 3D space. The final step of mesh reconstruction is taking the relationships inferred by contour and point correspondence, and generating triangles to connect points whilst taking these relationships into account.

Mackay \cite{mackay2019robust} begins with two ordered sets of points X and Y, from the two contours matched, with some edges provided by point correspondence. As a result of the constraints on DTW, there are three cases for each point $x_{m}$ on the first contour:
\begin{itemize}
\item $x_{m}$ has an edge with $y_{n}$ and $x_{m+1}$ has an edge with $y_{n+1}$. These points are direct neighbours on their respective contours, and form a quad which is trivial to triangulate.
\item $x_{m}$ has edges with a sequence of points $\{y_{n}, y_{n+1}, ..., y_{n+i}\}$. $x_{m}$ has a one-to-many point correspondence with these points. This can be triangulated with a triangle fan centering about $x_{m}$.
\item Each point in the sequence $\{x_{m}, x_{m+1}, ..., x_{m+i}\}$ has edges with a point $y_{n}$. This is the opposite of the previous case and can likewise be triangulated with a triangle fan centering about $y_{n}$.
\end{itemize}
Li et al. \cite{li2021method} observe that when m and n differ greatly, one-to-many point correspondences are common, and the triangulation becomes rough.

\subsection{The Branching Problem}

A simple structure is easy to infer. We see one contour on each slice the structure is present in, and in adjacent structures the contours are similarly positioned and shaped. More complex structures such as a branch show up differently, and it becomes harder to infer the original structure. Contour correspondence must be adjusted to deduce these unusual structures when they appear. Even when the general layout of the original structure is inferred through contour correspondence, point correspondence and mesh triangulation can be difficult when the shapes of contours change suddenly.

Mackay \cite{mackay2019robust} uses contour merging to help DTW point correspondence in the branching case. In a contour correspondence where there is a slice with a single contour and a slice with two contours, the two contours on the same slice are merged at their nearest point to give one larger contour. Point correspondence between the single contour and the merged contour can then proceed as normal. In some branching cases however, the triangulated mesh is twisted, as though DTW has not matched the correct points together. This problem is left for further research.

\subsection{Mesh Rendering}

Mackay \cite{mackay2019robust} states that surface rendering is not as widely implemented as volume rendering in medical contexts. However mesh rendering is abundant in literature from applications such as video games. Should mesh reconstruction become more accurate and faster, more implementations of mesh rendering for medical imaging may become available.

\section{Testing}

The end aim of these reconstructions and renders is to give humans a better visualisation of internal structures. However to compare methods objective testing is required. Requiring human experts to come to a consensus on desired outputs is resource consuming for stages such as image segmentation, and infeasible for complex structures such as meshes. Therefore metrics and repeatable methods have been devised to test algorithms.

\subsection{Measuring Similarity in Segmentation}

Pixel-wise XOR operations are common, simply working out where an ideal segmentation and an output segmentation differ.

Pu et al. \cite{pu2008adaptive} use a reference segmentation contour (defined by experts), and evaluate the distribution of distance error along their result contour from this reference. 

\subsection{Generating Models}

It is difficult to obtain real models of internal structures. Techniques to generate synthetic models suitable for testing have been developed.

Mackay \cite{mackay2019robust} uses Blender3D to create test models, by creating surface of revolutions about bezier curves. Multiple surfaces are merged for more complex structures such as branches. The main intent here was to create simple models of problem structures. Contours can be sampled from these models by plane intersections.

Mackay \cite{mackay2019robust} also proposes an alternative method of generating test data, where an image of a branching structure is drawn in black. An inter-slice distance is defined to separate rows. Each row is scanned for black pixels. When a black pixel is encountered, a contour is generated by revolving points about this pixel location. Noise can be added in. Where contours intersect, they are merged. This allows for branching structures to be generated by this method.

Pluta et al. \cite{pluta2012new} propose a rule-based method of generating lung models, including deformations and noise.

\subsection{Measuring Similarity in 3D Models}

Mackay \cite{mackay2019robust} uses Hausdorff distance (essentially a maximum deviation between point sets) to measure mesh similarity. Points are sampled from a ground truth model and the nearest distance is found to the reconstructed model.

\section{Parallelization}

Recent papers in medical imaging reconstructions tend to have an emphasis on improving performance through parallel processing. This is intended to take advantage of modern machines with more focus on core count rather than per core performance.

Saxena et al. \cite{saxena2013image} describe parallel techniques for various image processing tasks such as segmentation and noise reduction. The main idea is to tile an image and let different processors work on different tiles, then deal with the borders when recombining the tiles. They found their parallel implementation to be 2.5 times faster than a sequential implementation.


\section{Conclusion}

A variety of general approaches to reconstructing and rendering internal body structures are available, with many specific techniques investigated within each of them. Continued research and ironing out the flaws in these approaches will encourage their adoption in applications, enabling users to make more informed decisions in the field.

\bibliographystyle{acm}
\bibliography{references}

\end{document}