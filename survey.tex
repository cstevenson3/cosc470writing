\documentclass[acmsmall]{acmart}

\begin{document}

\title{A Survey of Medical Imaging Reconstructions and Rendering}
\author{Cameron Stevenson}
\affiliation{
 \institution{University of Canterbury}
 \streetaddress{Ilam}
 \city{Christchurch}
 \state{Canterbury}
 \country{New Zealand}}
 
\begin{abstract}
Abstract a b c
\end{abstract}

\maketitle

\section{Introduction}

Intro a b c

\section{Overview}

Overview a b c

\section{Image Processing}

\subsection{Segmentation}

Birkfellner \cite{birkfellner2016applied} observes that organs are usually composed of multiple tissue types, which show up as different intensities under imaging. This makes segmentation "a rather complex, specialized procedure often requiring considerable manual interaction". Particular organs are focussed on when developing segmentation methods.

Birkfellner \cite{birkfellner2016applied} covers some advanced segmentation methods. The watershed transform for example uses the physical idea of water running to the bottom of valleys in a landscape. After taking a gradient transform on an image, edges are peaks in the landscape, and the virtual water will fill up basins representing segments in the image. Various interpretations of the physical behaviour can be used.

Mukundan \cite{mukundan2016reconstruction} observes that in HRCT lung scans, tissue regions are "characterized by different and easily separable intensity levels". Simple thresholding can be used to pick out regions. 

\subsection{Contour Finding}

Mukundan \cite{mukundan2016reconstruction} starts with a binary image after thresholding. Eroding the image with a 3x3 element then subtracting this from the thresholded image gives one pixel wide edges. Sequential edge following is used to extract contours. Discarding small contours reduces the number of contours significantly.

Mackay \cite{mackay2019robust} uses contours generated from Adaptive Contour Marching.

\section{Volumetric Rendering}

Maximum intensity projection (MIP) is a raycasting method where rays project the most intense voxel they pass through \cite{birkfellner2016applied}. The images produced have high contrast detail and are easy to understand.

Summed voxel rendering is another raycasting method where rays sum up intensities from every voxel they pass through, giving a blurred image \cite{birkfellner2016applied}.

Intersecting arbitrary rays with voxels can be computationally expensive. Shear-warp rendering solves this by using projections which make rays orthogonal to the voxel axes \cite{lacroute1994fast}. 


\section{Surface Reconstructions}

\subsection{Point Cloud Methods}


\subsection{Contour Correspondence}
contour correspondence...

\subsection{Point Correspondence}
Point correspondence is an optional step in surface reconstructions, where points on matched contours are matched to each other as a precursor to triangulation.

Mackay \cite{mackay2019robust} proposes Dynamic Time Warping (DTW) as a method of point correspondence. DTW is intended to match features on the same structure across different times. In point correspondence, it matches points on contours which are from the same structure but in different slices, so slightly warped. 

\subsection{Mesh Triangulation}
mesh triangulation...

\subsection{The Branching Problem}

Mackay \cite{mackay2019robust} uses contour merging to help DTW point correspondence in the branching case. In a slice with one contour and a slice with two contours, the two contours are merged at their nearest point to give one larger contour. Point correspondence between the matched contours can then proceed as normal.

\subsection{Mesh Rendering}
mesh rendering...

\section{Testing}
\subsection{Generating Models}

Mackay \cite{mackay2019robust} uses Blender3D to create test models, by creating surface of revolutions about bezier curves. 

Pluta et al. \cite{pluta2012new} propose a rule-based method of generating models, including deformations and noise.

\subsection{Measuring Similarity}

Mackay \cite{mackay2019robust} uses Hausdorff distance (essentially a maximum deviation between point sets) to measure mesh similarity. Points are sampled from a ground truth model and the nearest distance is found to the reconstructed model.

\section{Conclusion}

\bibliographystyle{acm}
\bibliography{references}

\end{document}