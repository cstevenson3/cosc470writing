\documentclass[12pt]{article}
\usepackage[a4paper, portrait, margin=0.9in]{geometry}
\setlength{\parskip}{1em}

\title{Progress Report: Handling Structural Inconsistencies in CT Image Reconstructions}
\author{Cameron Stevenson\\[0.5cm]{\small Supervisor: Ramakrishnan Mukundan}}

\begin{document}
\maketitle

\section{Overview}

This is a progress report for a Computer Science Honours project related to mesh reconstructions from CT image stacks. I will recap the most relevant content of my project proposal and literature survey, and describe my progress in investigating improvements on prior methods. I will also take this opportunity to discuss techniques yet to be investigated and my rationale for them. Finally I will outline my goals and schedule for the remainder of the year.

\section{Introduction}

Medical imaging modalities such as CT scans and MRI produce stacks of images depicting regions of different tissue. These image stacks can be used to reconstruct the 3D structure of internal organs. The renders or models produced are useful in diagnosis, treatment, education, surgical simulation, and robot assisted surgery \cite{mackay2019robust, mukundan2016reconstruction}. The process of producing renders or reconstructions can be improved by reducing manual user input, improving reconstruction accuracy, or improving computational efficiency.

Branching structures can be problematic to reconstruct and are a common issue in prior methods. My project aims to improve the accuracy of reconstructions involving branches and other problematic structures.

\section{Background}

\subsection{Generic Methods}

The simplest form of rendering in medical imaging is direct volume rendering. The pixels in the image stacks are treated as voxels, forming a 3D grid of intensity data. Rasterisation or raycasting methods can be applied to these voxels to give a render. These renders are easy to understand and detailed, however they are computationally expensive.

Mesh rendering is preferred for real-time applications, but this requires a mesh reconstruction first. The mesh represents where boundaries of tissue lie in 3D space. To begin with segmentation of images must be done to find boundaries in 2D space. Segmentation methods are usually specific to an organ of interest and are tuned as such. A segmentation gives contours of points representing the boundaries between tissue. These points are then used for mesh reconstructions.

Point cloud methods for mesh reconstruction are adopted from applications like laser scanning, where a lot of points are dumped in as input and the algorithm has to infer which are connected. The contour points from segmentation can be used as input in a point cloud method to produce a reconstruction. Point cloud methods typically interpolate from points and don't use them directly in the output mesh.

Marching cubes is another method for mesh reconstruction. Boundary voxels are found, and their neighbouring voxels are considered in a lookup table to decide what local triangulation should be made for that voxel. All the triangles are then connected for the whole mesh.

\subsection{Correspondence Reconstructions}




\section{Progress}

Progress...

\section{Future}

Future...

\pagebreak
\bibliographystyle{IEEEtran}
\bibliography{references}

\end{document}
